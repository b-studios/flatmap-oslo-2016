\documentclass[compress]{beamer}
\usepackage{pgf}
\usepackage{color}
% \usepackage{minted}

% \renewcommand\texttt[1]{\mintinline{haskell}/#1/}

% \newminted{scala}{linenos}
% \newmintedfile{scala}{linenos,firstnumber=1}
% \newmint{scala}{}

\mode<presentation>

\useinnertheme{default}

\definecolor{Zeus}{RGB}{60,59,55}
\definecolor{MidnightBlue}{RGB}{68,72,169}
\definecolor{Supernova}{RGB}{227,158,0}
\definecolor{MediumSpringGreen}{RGB}{0, 227, 158}
\definecolor{Purple}{RGB}{158,0,227}
\definecolor{DarkOrange}{RGB}{227,101,0}
\definecolor{CCBlue}{RGB}{42,158,145}
\definecolor{shadecolor}{RGB}{227,158,0}

\usecolortheme[named=CCBlue]{structure}

\setbeamercolor{my header}{fg=CCBlue,bg=Zeus}
\setbeamercolor{my footer}{fg=CCBlue!90,bg=Zeus}
\setbeamercolor{my footer 2}{fg=CCBlue!90,bg=Zeus!90}
\setbeamercolor{my page number}{fg=black,bg=CCBlue!90}

\setbeamercolor{section in head/foot}{fg=white}
\setbeamercolor{mini frame}{fg=CCBlue}

\setbeamertemplate{headline}{%
  \begin{beamercolorbox}[wd=\paperwidth,ht=5ex,dp=5ex]{my header}
    \insertnavigation{\paperwidth}
  \end{beamercolorbox}
  \begin{pgfpicture}{0mm}{0mm}{0mm}{0mm}
    \pgfsetlinewidth{0.5mm}
    \color{CCBlue}
    \pgfline{\pgfpoint{0.01\paperwidth}{-1mm}}{\pgfpoint{0.8\paperwidth}{-1mm}}
  \end{pgfpicture}
  \begin{pgfpicture}{0mm}{0mm}{0mm}{0mm}
    \pgfsetlinewidth{0.5mm}
    \color{CCBlue}
    \pgfline{\pgfpoint{0.0025\paperwidth}{-0.76mm}}{\pgfpoint{0.0025\paperwidth}{-0.07\paperwidth}}
  \end{pgfpicture}
}

\setbeamertemplate{footline}{%
  \hbox{%
  \begin{beamercolorbox}[wd=0.55\paperwidth,ht=2.5ex,dp=1ex]{my footer}
    \hspace{1mm}\insertshortauthor{} - \insertshorttitle{}
  \end{beamercolorbox}
  \hspace{-2mm}
  \begin{beamercolorbox}[wd=0.40\paperwidth,ht=2.5ex,dp=1ex]{my footer 2}
    \hspace{1mm}\insertsubsection{}
  \end{beamercolorbox}
  \hspace{-2mm}
  \begin{beamercolorbox}[wd=0.06\paperwidth,ht=2.548ex,dp=1.01ex,center]{my page number}
    \insertframenumber{}
  \end{beamercolorbox}
}
  \begin{pgfpicture}{0mm}{0mm}{0mm}{0mm}
    \pgfsetlinewidth{0.5mm}
    \color{CCBlue}
    \pgfline{\pgfpoint{0.8\paperwidth}{4.5mm}}{\pgfpoint{0.99\paperwidth}{4.5mm}}
  \end{pgfpicture}
  \begin{pgfpicture}{0mm}{0mm}{0mm}{0mm}
    \pgfsetlinewidth{0.5mm}
    \color{CCBlue}
    \pgfline{\pgfpoint{0.985\paperwidth}{4.255mm}}{\pgfpoint{0.985\paperwidth}{0.1\paperwidth}}
  \end{pgfpicture}
}

\setbeamertemplate{navigation symbols}{}


\setbeamercolor*{block title}{fg=white,bg=Zeus}
\setbeamercolor*{block body}{bg=Zeus!20}

\setbeamercolor*{block title alerted}{use={normal text,alerted text},fg=black,bg=CCBlue}
\setbeamercolor*{block body alerted}{bg=Zeus,fg=Zeus!20}

\setbeamercolor*{block title example}{fg=black,bg=CCBlue!90}
\setbeamercolor*{block body example}{bg=Zeus!20}
\setbeamercolor*{example text}{fg=CCBlue}

\setbeamertemplate{mini frames}[box]
\setbeamersize{mini frame size=3pt}

\setbeamertemplate{blocks}[rounded][shadow=true]

% \usepackage{minted}
% \usemintedstyle{manni}

\title{Free Monads and Free Applicatives}

\author{Markus Hauck}

% \date{November 6, 2015}

\begin{document}

\begin{frame}
  \titlepage{}
\end{frame}

\section{Introduction}
\label{sec:Introduction}

\begin{frame}
  \frametitle{Writing DSLs}
  \begin{itemize}
  \item DSLs can be super handy
  \item popular way to write embedded DSL in Scala: Free Monads
  \item (less?) popular way: Free Applicatives
  \item this talk: use free monads and free applicatives
  \item discover the trade-offs
  \end{itemize}
\end{frame}

\begin{frame}
  \frametitle{Requirements}
  \begin{itemize}
  \item use functional programming library of your choice
  \item examples will use \texttt{cats}
  \item what you will see:
    \begin{itemize}
    \item DSL to speak with GitHub's API
    \item using free monads
    \item using free applicatives
    \item benefits of free applicatives
    \item using \textbf{both} and profit
    \end{itemize}
  \end{itemize}
\end{frame}

\begin{frame}
  \frametitle{The Goal of our DSL}
  \begin{itemize}
  \item offer interface to web API for GitHub
  \item let's keep it simple (or: I am lazy):
    \begin{enumerate}
    \item retrieve comment information from an issue
    \item get full name of a user from the login
    \end{enumerate}
  \end{itemize}
\end{frame}

\begin{frame}
  \frametitle{Starting with free monads}
  \begin{itemize}
  \item free monads offer easy way to write embedded DSLs
    \begin{itemize}
    \item write your instructions as an ADT
    \item use \texttt{Free} to lift them into the free monad
    \item use monadic interface to write programs
    \item interpret the resulting programs (possibly in many ways)
    \end{itemize}
  \item
  \end{itemize}
\end{frame}

\begin{frame}
  \frametitle{TODO}
\end{frame}

\end{document}